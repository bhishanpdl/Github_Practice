\documentclass[12pt]{article}
\usepackage[a4paper]{geometry}
\usepackage[myheadings]{fullpage}
\usepackage{fancyhdr}
\usepackage{lastpage}
\usepackage{graphicx, wrapfig, subcaption, setspace, booktabs}
\usepackage[T1]{fontenc}
\usepackage[font=small, labelfont=bf]{caption}
\usepackage{fourier}
\usepackage[protrusion=true, expansion=true]{microtype}
\usepackage[english]{babel}
\usepackage{sectsty}
\usepackage{url, lipsum}


\newcommand{\HRule}[1]{\rule{\linewidth}{#1}}
\onehalfspacing
\setcounter{tocdepth}{5}
\setcounter{secnumdepth}{5}

%-------------------------------------------------------------------------------
% HEADER & FOOTER
%-------------------------------------------------------------------------------
\pagestyle{fancy}
\fancyhf{}
\setlength\headheight{15pt}
\fancyhead[L]{ASTR 5202 CLASS PROJECT 2017}
\fancyhead[R]{Bhishan Poudel}
\fancyfoot[R]{Page \thepage\ of \pageref{LastPage}}
%-------------------------------------------------------------------------------
% TITLE PAGE
%-------------------------------------------------------------------------------

\begin{document}

\title{ \normalsize \textsc{ASTR 5202 PROJECT}
		\\ [2.0cm]
		\HRule{0.5pt} \\
		\LARGE \textbf{\uppercase{Feedback From Active Galactic Nuclei In The Process Of Galaxy formation}}
		\HRule{2pt} \\ [0.5cm]
		\normalsize  \vspace*{5\baselineskip}}

\date{\today}

\author{
		Submitted by:\\
		Bhishan Poudel\\
		Department of Physics and Astronomy\\ 
        Ohio University}
\newpage
\maketitle
\clearpage

\section*{Abstract}

There are a lots of unanswered questions and challenges in the field of cosmology. However, there have been incessant quest for the answers to these questions which are still unsolved. Many telescope projects have been assigned, some are still in the phase of construction, and others are planned for the future to deal with these problems. One of the cosmological questions concerns with the galaxy formation is that "what is the role of feedback from the AGN in galaxy formation"? This question, in itself, is not free from problems and requires meticulous attention. For example, if we ignore the feedback in star formation, it gives unacceptably large star formation rate which is hard to belive, thereby demanding the deeper understanding and the thorough knowledge of the subject. To understand the role of AGN in galaxy formation to some extent, in this report, I will talk about feedback from supermassive black holes, modes of AGN feedback, positive feedback from AGN, and, SMBH formation.
\clearpage

{\tiny }\section*{Outline}
\begin{itemize}
  \item Introduction to AGN
  \item Observational Evidence of AGN Feedback
  \item THE $ M - \sigma$ RELATION
  \item 8 SMALL vs LARGE SCALE FEEDBACK king
  \item The interaction between feedback from active galactic nuclei and supernovae website
 \end{itemize}
\clearpage

\section{Introduction to AGN}
When we look at sky in a clear night, we see that the sky is studded with stars of different brightness. If we again observe the same star some years later (not thosands of years) we do not see the much difference in their brightness. However, if human eyes were operating in radio wavelengths rather than optical band, the story will be totally different.In those cases,  If we observe the brightness pattern in a constellation, in a few years we would find that
some of the stars have become considerably brighter and others farily fainter than what had we observed few years ago. 

\clearpage
 \section{ Citations}
Following citations are used:

\cite{becerra} 
\cite{booth13}
\cite{car_ost06}

\cite{fabian12} 
\cite{ho08}
\cite{ish_fab12}  
  
\cite{king03}
\cite{king_pounds15}
\cite{sch_arny06}

\cite{silk_rees98}
\cite{silk_cintio_dvorkin13}
\cite{sp_gal07}

\bibliographystyle{plain}
\bibliography{ref}

\end{document}